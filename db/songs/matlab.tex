\begin{song}
\songtitle{Matlab}
\firstline{Jag har prövat nästan allt som finns att pröva på}
\alttitle{Man ska ha Matlab}

\begin{songmeta}
Alternativ titel: Man ska ha Matlab
Melodi: Husvagn
Text: Mattias Gustavsson, Tobias Lovnér, Olof Larsson
Musik: Galenskaparna \& After Shave
Sektioner: Data Chalmers
\end{songmeta}

\begin{songtext}
Jag har prövat nästan allt som finns att pröva på:
Beta, phaddrar, räknesticka, tärning eller så.
Jag har kalkylerat på de mest abnorma sätt
och nu så har jag kommit på hur man ska räkna rätt.

Man skall ha Matlab - då är kalkylen redan klar!
Man skall ha Matlab - det har jag sett att andra har!
Man skall ha Matlab - det är min livsfilosofi.
Man skall ha Matlab - för då blir man fri.

I många år så var jag inte alls så särskilt lärd.
Jag visste ej, vad väntar mig i denna stora värld.
Men sedan jag till Chalmers kom, och ända sedan dess
så har jag funnit livets stora lyxdelikatess.

Man skall ha Matlab - så att man slipper tänka alls.
Man skall ha Matlab - ja då går allting som en vals.
Man skall ha Matlab - där kan man rita sinusvåg.
Man skall ha Matlab - det låter som ett tåg.

\newpage
5 min vektorlab och 5 min matfys.
5 min kvantfysik och 5 min analys.
5 min fråga phadder 5 min fråga fop.
5 min tänka själv och sedan blir det stopp.

Man skall ha Matlab - och andas DD:s friska luft.
Man skall ha Matlab - det tycker tjejerna är tufft.
Man skall ha Matlab - när ryssen kommer med sitt MIG.
Man skall ha Matlab - då vinner man i krig.
\end{songtext}

\begin{songnotes}
Förklaring till \textquotedblleft{}\ldots då vinner man i
krig.\textquotedblright{}: Christer Borell har en gång sagt att
\textquotedblleft{}Det är så det är i krig nu för tiden. Det är den som räknar
ut fouriertransformerna snabbast som vinner.\textquotedblright{} Och då måste
det ju vara så!
Förklaring till \textquotedblleft{}\ldots det låter som ett tåg.%
\textquotedblright{}: \\ Varje Matlabanvändare bör någon gång i sitt liv ha
skrivit load train följt av sound(y) och njutit av resultatet.
De som sitter i DD (datorsalen på F, Chalmers) behöver bara skriva matlabmusik
för att sången skall spelas upp.
\end{songnotes}
\end{song}
