\begin{song}
\songtitle{Sång till Norden}

\alttitle{Du gamla, Du fria}
\alttitle{Till svenska fosterjorden}
\begin{songmeta}
Alternativ titel: Du gamla, Du fria
Alternativ titel 2: Till svenska fosterjorden
Melodi: Sång till Norden
Text: Richard Dybeck
Musik: Trad.
\end{songmeta}

\begin{songtext}
Du gamla, Du fria, Du fjällhöga nord,
Du tysta, Du glädjerika sköna!
Jag hälsar Dig, vänaste land uppå jord,
||: Din sol, Din himmel, Dina ängder gröna. :||

Du tronar på minnen från fornstora da'r,
då ärat Ditt namn flög över jorden.
Jag vet att Du är och Du* blir vad Du var.
||: Ja, jag vill leva, jag vill dö i Norden. :||
\end{songtext}

\begin{songnotes}
Till skillnad från många andra nationalsånger har denna sång aldrig officiellt, genom något politiskt beslut, antagits som nationalsång. Visan framfördes för första gången den 13 november 1844, och publicerades i tryck 1865 i tidskriften Runa.
Ytterligare två verser, med mer uttalat svenskt tema, skrevs av Louise Ahlén, men sjungs mycket sällan och har därför heller inte tagits med här.
*: \textquotedblleft{}Du blir\textquotedblright{} i andra versen är i original endast \textquotedblleft{}blir\textquotedblright{}, men \textquotedblleft{}Du blir\textquotedblright{}, eller \textquotedblleft{}förblir\textquotedblright{} sjungs oftast och passar även bättre i melodin.
\end{songnotes}

\end{song}
