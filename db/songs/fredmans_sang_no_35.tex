\begin{song}
\songtitle{Fredmans sång no. 35}
\alttitle{Gubben Noak}
\begin{songmeta}
Alternativ titel: Gubben Noak
Melodi: Fredmans sång no. 35 (Carl Michael Bellman)
Text: Carl Michael Bellman
\end{songmeta}

\begin{songtext}
Gubben Noak, gubben Noak
var en hedersman.
När han gick ur arken,
plantera' han på marken
mycket vin, ja mycket vin, ja
detta gjorde han.

Noak rodde, Noak rodde
ur sin gamla ark.
Köpte sig buteljer,
sådana man säljer
för att dricka, för att dricka
på vår nya park.

Han väl visste, han väl visste
att en män'ska var
törstig av naturen,
som de andra djuren.
Därför han ock, därför han ock
vin planterat har.

Gumman Noak, gumman Noak
var en hedersfru:
Hon gav man sin dricka.
Fick jag sådan flicka,
gifte jag mig, gifte jag mig
just på stunden nu.

Aldrig sa hon, aldrig sa hon
\textquotedblleft{}Kära far, nånå,
sätt ifrån dig kruset!\textquotedblright{}
Nej, det ena ruset
på det andra, på det andra
lät hon gubben få.

Gubben Noak, gubben Noak
brukte egna hår,
pipskägg, hakan trinder,
rosenröda kinder.
Drack i botten, drack i botten.
Hurra och gutår!

Då var lustigt, då var lustigt
på vår gröna jord:
Man fick väl till bästa
ingen törstig nästa
satt och blängde, satt och blängde,
vid ett dukat bord.

Inga skålar, inga skålar
gjorde då besvär.
Då var ej den läran
\textquotedblleft{}Jag skall ha den äran!\textquotedblright{}
Nej, i botten, nej, i botten
drack man ur så här.
\end{songtext}
\end{song}
