\begin{song}

\songtitle{Det var i vår ungdoms fagraste vår}
\alttitle{Årskursvisan}
\begin{songmeta}
Melodi: Trad.
\end{songmeta}

\begin{songtext}
Det var i vår ungdoms fagraste vår,
vi drack varandra till och vi sade gutår!
Alla så dricka vi nu D-Osquarulda till.
Och D-Osquarulda vi säger inte nej därtill.
För det var i vår ungdoms fagraste vår,
vi drack varandra till och vi sade gutår!
\end{songtext}

\begin{songnotes}
D-Osquarulda utbytes lämpligen mot årskursnamn, \\
\textquotedblleft{}gästerna\textquotedblright{} och eventuellt \textquotedblleft{}personalen\textquotedblright{}.

Vartefter Konglig Datasektionens årskurser fick längre och längre namn, innebar
det en större utmaning att hinna sjunga \textquotedblleft{}Och alla så dricka vi nu\ldots\textquotedblright{}.
Därför föreslås följande sätt att sjunga längre-än-tvåstaviga namn:
\begin{itemize}
\setlength{\itemsep}{0cm}
\setlength{\parskip}{0cm}
\item \textquotedblleft{}Och alla vi dricka D-Osquarulda till\textquotedblright{} \\(Ex: dovicesimus)
\item \textquotedblleft{}Och alla, drick D-Osquarulda till\textquotedblright{} \\(Ex: dodevicesimus)
\item \textquotedblleft{}Och alla, nu drick D-Osquarulda till\textquotedblright{} \\(Ex: vicesimus quartus)
\end{itemize}

\newpage
\paragraph{Årskursnamn hittills}
\rule{\textwidth}{0pt}
\begin{multicols}{2}
\begin{enumerate}
\setcounter{enumi}{1982}
\setlength{\itemsep}{0cm}
\setlength{\parskip}{0cm}
\item primus
\item secundus
\item tertius
\item quartus
\item quintus
\item sextus
\item septimus
\item octavus
\item nonus
\item decimus
\item undecimus
\item dodecimus
\item tertius decimus
\item sigvard
\item quintus decimus
\item sextus decimus
\item septus decimus
\item dodevicesimus
\item undevicesimus
\item vicesimus
\item unvicesimus
\item dovicesimus
\item tertius vicesimus
\item vicesimus quartus
\item vicesimus quintus
\item vicesimus sextus
\item vicesimus septus
\item duodetricesimus
\item undetricesimus
\item tricesimus
\item untricesimus
\end{enumerate}
\end{multicols}
\end{songnotes}

\end{song}
