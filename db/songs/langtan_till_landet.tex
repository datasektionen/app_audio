\begin{song}
\songtitle{Längtan till landet}
\alttitle{Vintern rasat\ldots}

\begin{songmeta}
Alternativ titel: Vintern rasat\ldots
Melodi: Längtan till landet
Text: Herman Sätherberg
Musik: Otto Lindblad
\end{songmeta}

\begin{songtext}
Vintern rasat ut bland våra fjällar,
drivans blommor smälta ner och dö.
Himlen ler i vårens ljusa kvällar,
solen kysser liv i skog och sjö.

||: Snart är sommar'n här i purpurvågor,
guldbelagda, azurskiftande,
ligga ängarne i dagens lågor,
och i lunden dansa källorne. :||

Ja, jag kommer! Hälsen glada vindar
ut till landet, ut till fåglarne,
att jag älskar dem, till björk och lindar,
sjö och berg, jag vill dem återse,

||: se dem än som i min barndoms stunder,
följa bäckens dans till klarnad sjö,
trastens sång i furuskogens lunder,
vattenfågelns lek kring fjärd och ö. :||
\end{songtext}
\newpage
\begin{songnotes}
Texten publicerades ursprungligen som vers i boken Jägarens vila, 1838. Några år
efter att hans gode vän satt musik till versen skrev Herman Sätherberg ny text
till visan. Båda dessa textversioner börjar ungefär likadant. Ursprunglig
version är 4 verser lång, den reviderade versionen är 6 verser lång. Det är de
två första verserna i den reviderade versionen som blivit folkkära och som
återges här.
Det \textquotedblleft{}land\textquotedblright{} textförfattaren längtade till
när texten skrevs var Nolinge säteri i Grödinge socken, \\
Södermanland.
\end{songnotes}

\end{song}
